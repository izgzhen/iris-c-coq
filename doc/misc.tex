\section{Misc}
\label{sec:misc}

This section contained some key formal developments claimed but not yet mechanized in Coq for reference.

\subsection{Properties of Evaluation Context}

Our expression $\Expr$ is a recursively defined algebraic data type, which can be generalized into the following form:

\[
e : \Expr \bnfdef{} \Expr_1(A_1, e^{r_1}) \mid ... \mid \Expr_n(A_n, e^{r_n}) \mid v
\]

Here, $\Expr_i$ is tag for $i$-th class of expression, $A_i$ is its arbitrary non-recursive payload, and $e^{r_i}$ means
it has $r_i \in \nat$ recursive occurences.

It is apparent that $\ectx$ is a direct translation of $\Expr$ plus some ordering considerations, though in actual Coq development we don't mechanize this fact.
But we will make it explicit here to support some stronger claims than what we can do in Coq.

For some abstract $\Expr$ like above, its $\ectx$ should be a sum of sub-$\ectx_i$ for each $\Expr_i$. When $r_i = 0$, $\ectx_i$ doesn't exist, now we consider
$r_i > 0$, define

\[
k_i: \ectx_i \bnfdef{} \ectx_{i1}(A_i, e^{r_i - 1}) \mid \ectx_{i2}(A_i, v^1, e^{r_i - 2}) \mid ... \mid \ectx_{ir_i}(A_i, v^{r_i - 1})
\]

\begin{theorem}
For any $e, e': \Expr$ and $k_{im}, k_{jn}: \ectx$,
\[k_{im}(e) = k_{jn}(e') \proves i = j \]
\end{theorem}
\begin{proof} Trivial. \end{proof}

\begin{theorem}
For any $e, e': \Expr$ and $k_{im}, k_{in}: \ectx_i$,
\[k_{im}(e) = k_{in}(e') \proves
 (m = n \land e = e') \lor
 (m \neq n \land (\Exists v. \toval(e) = v \lor \Exists v. \toval(e) = v))\]
\end{theorem}
\begin{proof} 

When $m = n$, by injectivity;
When $m \neq n$, we can expand the equation like below without loss of generality:
\begin{align*}
\Expr_i(A_i, v_1, ..., v_{m - 1}, &e, e_{m + 1}, ..., e_{r_i - 1}) = \\
\Expr_i(A_i', v_1', ..., v_{m - 1}', &v_m', ..., e', ...)
\end{align*}
So $e = v_m'$ by injectivity.
\end{proof}

As you may observe, the expression space spanned by $k(e)$ for any $k, e$ is not the entire
expression space as an ADT. Instead, it is a subset of $\Expr$ which represents well-formed ones
that can appear as an immediate form according to some well-defined evaluation order. We call
such $e$ is \emph{well-formed}.

Here we prove two lemmas about the syntactic structure on paper:

\begin{lemma}
$\All e: \Expr, k: \cont . \isenf(e) \ra \unfill(k(e)) = (k, e)$
\end{lemma}
\begin{proof}

Let's prove inductively w.r.t $k$. When $k$ is empty,
since $e$ is in normal form, so unfill it will only return the same thing.
And inductively, since fill and unfill should cancel out, the final conclusion is proved trivially as well.

\end{proof}

\begin{lemma}
Induction scheme for $\Expr$:
\begin{align*}
\All P: \Expr \ra \Prop.
  &(\All e. \isenf(e) \ra P(e)) \ra \\
  &(\All e, k: \cont . \toval(e) = \bot \ra \wellformed(e) \ra P(e) \ra P(k(e))) \ra\\
  &(\All e. \toval(e) = \bot \ra \wellformed(e) \ra P(e))
\end{align*}
\end{lemma}
\begin{proof}
We want to \emph{inductively} prove that for any non-value, \emph{well-formed} $e$, $P(e)$ holds.
Note that $P$ here can be any proposition.

The base case is when $e$ is in normal form, which corresponds to the first condition;
The inductive case is that for any well-formed $e'$ that is not in normal form,
it must be of the form $k(e)$ for some $k, e$. With each proved given sufficient inductive
assumption, we know that any well-formed $e$ must satisfy $P$.
\end{proof}
