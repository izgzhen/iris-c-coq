\section{Extensions}\label{sec:extension}


Besides the core language and its logic,
we built several extension on top of it in a compatible way.

\subsection{Array}

Using the most primitive form of aggregated structure, product type, we can easily derive
array (implemented) or even \texttt{struct} in future work.

``A pointer $p$ pointing to an array $vs$ of $n$ elements which are $\tau$-typed'' is expressed with:
\[p \mapsto_q vs: \tau^n\]

In implementation, $vs$ is \texttt{varray (l:list val)}, and $\tau^n$ is \texttt{tyarray (t:type) (n:nat)}.

We also have an intermediate representation of a slice of array:
\[p \mapsto_q [v_i : \tau, ..., v_{i + l - 1}:\tau]\]

The first form is easier for allocation, since it is defined as a unified aggregate structure;
but the second form is easier for use, for example, splitting or indexing. We have the following lemmas proven in Coq:

\begin{mathpar}
\infer[split-slice]{}
{p \mapsto_q [v_i : \tau, ..., v_{i + l_1 - 1}:\tau] * p \mapsto_q [v_{i + l_1} : \tau, ..., v_{i + l_1 + l_2 - 1}:\tau]
 \equiv p \mapsto_q [v_i : \tau, ..., v_{i + l_1 + l_2 - 1}:\tau]}

\infer[index-spec]{i < n}
{\hoare{p \mapsto_q vs: \tau^n}{\edereft{\tau}{p + i}}
       {\Ret v. p \mapsto_q vs: \tau^n * \lookupmap{vs}{i} = \Some{v}}}

\end{mathpar}
