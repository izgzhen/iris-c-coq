\section{Language}
\label{sec:language}

\subsection{Definitions}

\paragraph{Syntax.}\label{p:type}

The language is a simplified version of C. It mainly consists of
\Expr{} (expressions), \Type{} (types) and \Val{} (values).

\begin{align*}
    \tau : \Type \bnfdef{}&
        \tyvoid \mid
        \tynull \mid
        \tybyte \mid
        \tyword \mid
        \typtr{\tau} \mid
        \tau \times \tau
\\
    v : \Val \bnfdef{}&
        \void \mid
        \vnull \mid
        i \in [0, 2^8) \mid
        i \in [0, 2^{32}) \mid
        l \mid
        (v, v)
\\
    e : \Expr \bnfdef{}&
       v \mid
       x \mid
       e \oplus e \mid
       \edereft{\tau}{e} \mid
       \eaddrof{e} \mid
       \eCAS{\tau}{e}{e}{e} \mid
       \efst{e} \mid
       \esnd{e} \mid
       (e, e) \mid
       \\&
       e \la e \mid
       \eif{e}{e}{e} \mid
       \ewhile{e}{e} \mid
       \ebreak \mid
       \econtinue \mid
       \\&
       \erete{e} \mid
       \ecall{\tau}{f}{e_1, (..., e_n)} \mid
       e ;; e \mid
       \ealloc{\tau}{e} \mid
       \efork{\tau}{f}
\\
\end{align*}

We also have following definitions:

\begin{itemize}
  \item block address $b \in \nat$
  \item offset $o \in \integer^{+}$
  \item Memory address $l: \Addr \eqdef \vaddr{b}{o}$
  \item Program $\Gamma: \Text \eqdef [f \gmapsto \Funct]$ is a list of functions indexed by names
  \item Function $F: \Funct \eqdef (\tau_{ret} \times [x_1 : \tau_1, ...] \times e)$ consists of return type,
    parameter declarations and function body.
\end{itemize}

(NOTE) Compared with C, there are several notable differences:
\begin{enumerate}
  \item We currently don't support \texttt{for} loop and \texttt{switch}
  \item C differentiates statements and expressions (so do many other program logics). For simplicity and
    expressivity, we merged them together into a coherent expression $\Expr$ instead
  \item We support declarations of local variables through a syntactic sugar based on the core language
        (definition of $\mathrm{default}: \Type \ra \Val$ is trivial)
        \[ \edecl{\tau}{x}{e} \eqdef \elet{\tau}{x}{\ealloc{\tau}{\mathrm{default}(\tau)}}{e}\]
\end{enumerate}

(NOTE) We sometimes use $\texttt{c} : \Expr$ to denote \emph{condition} expression
       in \texttt{if} and \texttt{while}.

\paragraph{Evaluation Context.}

To make the evaluation order explicit and reflected to the logic through the bind rule,
we define the \emph{evaluation context}:

\begin{align*}
    \ctx : \ectx \eqdef{}&
        \emptyctx \oplus e \mid
        v \oplus \emptyctx \mid
        \edereft{\tau}{\emptyctx} \mid
        \eaddrof{\emptyctx} \mid
        \efst{\emptyctx} \mid
        \esnd{\emptyctx} \mid
        (\emptyctx, e) \mid
        (v, \emptyctx) \mid
        \\&
        \eCAS{\tau}{\emptyctx}{e}{e} \mid
        \eCAS{\tau}{l}{\emptyctx}{e} \mid
        \eCAS{\tau}{l}{v}{\emptyctx} \mid
        \\&
        \emptyctx \la e \mid
        l \la \emptyctx \mid
        \eif{\emptyctx}{e}{e} \mid
        \\&
        \erete{\emptyctx} \mid
        \ecall{\tau}{f}{\emptyctx} \mid
        \ealloc{\tau}{\emptyctx} \mid
        \emptyctx ; e
\end{align*}

\subsection{Semantics}\label{sec:semantics}
\paragraph{Model.}

First define
\begin{itemize}
\item Byte-size value \[\bval : \byteval \eqdef \mundef \mid i \in [0, 2^8) \mid l_{\{0 \mid 1 \mid 2 \mid 3\}} \mid \vnull\]
\item Heap $h: \Heap \eqdef [ l \gmapsto \bval ]$
\item Text $\Gamma: \Text \eqdef [ f \gmapsto \Funct ]$
\item Continuation $k: \Cont \eqdef [\ectx]$
\item Context $K: \Ctx \eqdef \kwhile{c}{e}{k} \mid \kcall{k}{\epsilon}$
\item Stack $s: \Stack \eqdef [\Ctx]$
\item Environment: $ \epsilon : [x \gmapsto (\tau, v)]$
\item Shared state \[\sigma: \State \eqdef (\text{heap}: \Heap, \text{text}: \Text)\]
\item Local state \[\sigma_l: \LocalState \eqdef (\text{stack}: \Stack, \text{env}: \Env)\]
\end{itemize}

(NOTE) You may notice the difference between $\Val$ and $\byteval$, it is intended to create a layer of abstraction for
easier manipulation of spatial assertions and also for a clean, unified language syntax.

\paragraph{Small-Step Operational Semantics.}

We define the small-step semantics (see figure \ref{fig:semantics}) for three sub-semantics:

\begin{itemize}
    \item local reduction $\estepf{\Gamma}{\epsilon}{e}{h}{\epsilon'}{e'}{h'}{t_1, t_2, ...}$
          (when the set of forked-off threads $t_1, ...$ is empty, we abbreviate it as
          $\estep{\Gamma}{\epsilon}{e}{h}{\epsilon'}{e'}{h'}$)
    \item \emph{jump} reductions $\jstep{\Gamma}{\epsilon}{e}{s}{\epsilon'}{e'}{s'}$
    \item \emph{while} reductions $\wstep{e}{s}{e'}{s'}$
\end{itemize}

Then we combine them together into $\cstepf{\Gamma}{\epsilon}{e}{s, \sigma}{\epsilon}{e'}{s', \sigma'}{t_1, ...}$ point-wise.

\begin{figure}[!ht]
\begin{mathpar}\label{fig:semantics}

\infer[es-binop]{\llbracket oplus \rrbracket (v_1, v_2) = v'}{\estep{\Gamma}{\epsilon}{v_1 \oplus v_2}{h}{\epsilon}{v'}{h}}

\infer[es-deref-typed]{\tychk{v}{\tau} \quad h \vdash l \mapsto \encode(v)}{\estep{\Gamma}{\epsilon}{\edereft{\tau}{l}}{h}{\epsilon}{v}{h}}

\infer[es-fst]{}{\estep{\Gamma}{\epsilon}{\efst{(v_1, v_2)}}{h}{\epsilon}{v_1}{h}}

\infer[es-snd]{}{\estep{\Gamma}{\epsilon}{\esnd{(v_1, v_2)}}{h}{\epsilon}{v_2}{h}}

\infer[es-assign]{}{\estep{\Gamma}{\epsilon}{l \la v}{h}{\epsilon}{\eskip}{h[ l \mapsto \encode(v) ]}}

\infer[es-seq]{}{\estep{\Gamma}{\epsilon}{v ;; e}{h}{\epsilon}{e}{h}}

\infer[es-alloc]{\tychk{v}{\tau} \quad \All o'. h(b, o') = \bot}{\estep{\Gamma}{\epsilon}{\ealloc{\tau}{v}}{h}{\epsilon}{\vaddr{b}{o}}{h[\vaddr{b}{o} \mapsto v]}}

\infer[es-if-true]{}{\estep{\Gamma}{\epsilon}{\eif{\vtrue}{e_1}{e_2}}{h}{\epsilon}{e_1}{h}}

\infer[es-if-false]{}{\estep{\Gamma}{\epsilon}{\eif{\vfalse}{e_1}{e_2}}{h}{\epsilon}{e_2}{h}}

\infer[es-bind']{\isjmp(e) = \FALSE \quad \estep{\Gamma}{\epsilon}{e}{h}{\epsilon'}{e'}{h'}}{\estep{\Gamma}{\epsilon}{(k::ks)e}{h}{\epsilon'}{(k::ks)e'}{h'}}

\infer[es-cas-fail]
{\tychk{v_1}{\tau} \quad \tychk{v_2}{\tau} \quad \tychk{v}{\tau} \quad h \vdash l \mapsto \encode(v) \quad v_1 \neq v}
{\estep{\Gamma}{\epsilon}{\eCAS{\tau}{l}{v_1}{v_2}}{h}{\epsilon}{\vfalse}{h}}

\infer[es-cas-suc]
{\tychk{v_1}{\tau} \quad \tychk{v_2}{\tau} \quad h \vdash l \mapsto \encode(v_1)}
{\estep{\Gamma}{\epsilon}{\eCAS{\tau}{l}{v_1}{v_2}}{h}{\epsilon}{\vtrue}{h[ l \mapsto \encode(v_2) ]}}

\infer[es-fork]
  {\Gamma(f) = \Funct(\tau, [], e)}
  {\estepf{\Gamma}
          {\epsilon}{\efork{\tau}{f}}{h}
          {\epsilon}{\void}{h}{\{e[ps/vs]\}}}

\infer[js-rete]
  {\All K \in KS'. K \neq \kcall{\any}}
  {\jstep{\Gamma}{\epsilon}{k'(\erete{v})}{KS' \append \kcall{k} :: KS}{\epsilon}{k(v)}{KS}}

\infer[js-call]
  {\Sigma(f) = \Funct(\tau, ps, e)}
  {\jstep{\Gamma}{\epsilon}{k(\ecall{\tau}{f}{vs})}{KS}{\epsilon}{e[ps/vs]}{\kcall{k}::KS}}

\infer[ws-while]
  {}
  {\wstep{k(\ewhile{c}{e})}{KS}{\eif{c}{e;\econtinue}{\ebreak}}{\kwhile{c}{e}{k} :: KS}}

\infer[ws-break]
  {\All K \in KS'. K \neq \kwhile{\any}{\any}{\any}}
  {\wstep{k'(\ebreak)}{KS' \append \kwhile{c}{e}{k} :: KS}{k(\void)}{KS}}

\infer[ws-continue]
  {\All K \in KS'. K \neq \kwhile{\any}{\any}{\any}}
  {\wstep{k'(\econtinue)}{KS' \append \kwhile{c}{e}{k} :: KS}{k(\ewhile{c}{e})}{KS}}

\end{mathpar}
\caption{Semantics rules}
\end{figure}

\subsection{Type System and Lexical Environment}

\paragraph{Local Typing Rules.}

The types are defined in \Sref{p:type}, and all values in $\val$ can be \emph{locally} typed trivially (since $\val$ is introduced to reflect type structure in some sense). Nevertheless, due to the fact that the language of study is weakly-typed in the vein of C, we still have some ``weird'' rules worth documenting.

Here we define local typing judgment $\tychk{v}{\tau}$ for values.

\begin{mathpar}
\infer[typeof-void]{}{\tychk{\void}{\tyvoid}}

\infer[typeof-null]{}{\tychk{\vnull}{\tynull}}

\infer[typeof-int8]{}{\tychk{i \in [0, 2^8)}{\tybyte}}

\infer[typeof-int32]{}{\tychk{i \in [0, 2^{32})}{\tyword}}

\infer[typeof-null-ptr]{}{\All \tau. \tychk{\vnull}{\typtr{\tau}}}

\infer[typeof-ptr]{}{\All \tau, l. \tychk{l}{\typtr{\tau}}}

\infer[typeof-prod]{\tychk{v_1}{\tau_1} \quad \tychk{v_2}{\tau_2}}{\tychk{(v_1, v_2)}{\tau_1 \times \tau_2}}

\end{mathpar}

Note that rule TYPEOF-NULL-PTR means that $\vnull$ can be of any pointer type,
and TYPEOF-PTR means that a pointer can have \emph{any} pointer type.

\paragraph{Lexical Environment}\label{par:tyev}

Formally, the lexical environment for a function body consists of two parts:
parameter bindings (free variables) and local bindings (introduced by \texttt{let}).

We model the C semantics that these two types of bindings are essentially the same,
except that we initialize the parameter bindings by passed in nested value pair
(note that it is in the value domain, rather than a Coq-level list, which simplifies the verification),
and initialize the local bindings by placeholder values.
